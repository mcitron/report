% Chapter 3

\chapter{Trigger} % Main chapter title

\label{Chapter3} % Change X to a consecutive number; for referencing this chapter elsewhere, use \ref{ChapterX}

\lhead{Chapter 3. \emph{Trigger}} % Change X to a consecutive number; this is for the header on each page - perhaps a shortened title

%----------------------------------------------------------------------------------------
%	SECTION 1
%----------------------------------------------------------------------------------------

\section{Trigger}
\subsection{Legacy System}
The Level 1 trigger (L1) is designed to filter from $40MHz$ to $\mathcal{O}(100kHz)$ to allow processing by the high level trigger (HLT) while keeping high acceptance for interesting events. This has worked well at previous luminosities up to $0.7\times10^{34} cm^{-2}s^{-1}$. However, with the upgrade the maximum luminosity will increase to $1.6\times10^{34} cm^{-2}s^{-1}$. The L1 trigger system has no tracking information so must rely on information from the electromagnetic (ECAL) and hadronic (HCAL) calorimeters only\cite{gct}. The trigger system takes input from  trigger towers (TT) corresponding to $5\times5$ ECAL crystals \footnote{Each TT corresponds to $0.087\times0.087$ in $\Delta\eta\times\Delta\phi$. For simplicity $\eta$ and $\phi$ units of one trigger tower are used in the remainder of this report.} and an identical area in the HCAL. The extent of the calorimeter is $56\times72$ The legacy Global Calorimeter Trigger (GCT) is designed to find jets, electrons and photons and to compute global energy sums. Trigger towers are grouped into $4\times4$ and features are found using algorithms described in \cite{algos}. Due to hardware limitations the detector is split into 16 regions which are processed by the Regional Calorimetric Trigger (RCT) to form candidates and sums energies. The RCTs must share information to account for features which occur at boundaries between them. The RCTs are then combined in the GCT which computes global quantities and sorts the jets before deciding on whether to store the event. The hardware and algorithms used by this system will be upgraded for the beginning Run 2 to form the Upgrade Calorimetric Trigger (UCT) \cite{UCT}.

To process the data with global algorithms there is a trade off between granularity and the sharing of data. An alternative to this is the Time Multiplexed Trigger (TMT)\cite{rose}. This splits the data by time rather than detector region. The data are buffered over several bunch crossings which allows the whole event to be processed on one board without loss of granularity. The results shown below concern studies for optimal jet algorithms using a trigger based on such a system. This the stage 2 trigger upgrade and will be run alongside the UCT in the commisioning stage and will become the trigger in 2016. 