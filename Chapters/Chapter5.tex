% Chapter Template

\chapter{Jets and PU at stage 2} % Main chapter title

\label{Chapter 5} % Change X to a consecutive number; for referencing this chapter elsewhere, use \ref{ChapterX}

\lhead{Chapter 5. \emph{Conclusions}} % Change X to a consecutive number; this is for the header on each page - perhaps a shortened title

%----------------------------------------------------------------------------------------
%	SECTION 1
%----------------------------------------------------------------------------------------

\section{Jet Algorithm}
\label{algo}
The focus of the results in this report will be on studies for jet definitions and pile-up subtraction algorithms for the stage 2 trigger. The legacy jet algorithm is described in \cite{gctalgo}. At stage 2 the TMT allows the granularity to be significantly increased compared to the legacy system (the cells used to make the jets will reduce from $4\times4$ towers to single towers). The algorithm must change to take advantage of this. Additionally In run to the increase in energy to $13TeV$ will mean jets are  boosted and therefore smaller in size. To account for this at level one the cone size should be decreased. The algorithm presented here uses a cone size of $9\times9$ (compared to $12\times12$ for the GCT). This corresponds to the decrease in jet size for the HLT - from $R=0.5$ to $R=0.4$ for the anti $k_T$ jet \cite{Antik_t}. Jets from the primary vertex are likely to be boosted and thus most of their energy is in the central (seed) tower. The algorithm is thus as follows: each trigger tower in turn is taken as the candidate. The towers in the 4 surrounding rings (or up to the edge in $\eta$) are then compared to the seed tower using the mask shown in figure \ref{mask}. If the comparison is true for any tower then the seed is vetoed and the next tower is checked. The mask is designed to be unambiguous for towers of equal energy. If a seed is not vetoed then a jet is defined at that tower position with $p_t$ given by the total energy of the towers.   
\begin{figure}
\centering
    \includegraphics[width=0.5\textwidth]{Figures/mask.png}
  \caption{Nine by nine mask used to define jets}
  \label{mask}
\end{figure}
The two main problems with this algorithm are that sufficiently close jets will appear as a single jet and that energy may be lost if a tower that vetoes another is itself vetoed by a tower sufficiently far away not to include the far tower's energy. The first case is common to all jet algorithms with a fixed cone size but as the energy is not lost the single jet or total energy will almost certainly be enough to trigger. Despite this a quality bit adapted from the idea of n-subjetiness \cite{nsub} is under investigation. The lost energy from chain vetos is found to be a small (~0.1\%) inefficiency.

\section{Pile Up Subtraction}
\subsection{Pile Up}
The luminosity of the LHC will be increased in the upgrade to $L = 1\times10^{34} cm^{-2}s^{-1}$. The easiest way to increase luminosity while maintaining stability is to use large, but widely separated, proton bunches \cite{PUform}. However, for the detector this means increased number of primary vertices (pile up) at the collision point. The average number of vertices per bunch crossing is given in equation \ref{pu}.

\begin{equation}
\langle N_p \rangle = \sigma L \tau_b
\label{pu}
\end{equation}

where $\langle N_p \rangle$ is the average number of vertices, $\sigma$ is the cross section and $\tau_b$ is the bunch spacing. At the end of run 1 with $8 TeV$ ($\sigma = 71.5mb$),$\tau_b = 50ns$ and $L = 0.75\times10^{34} cm^{-2}s^{-1}$ gave $\langle N_p \rangle\simeq27$. After LS1 the luminosity and $\sigma$ gain (to $76mb$) this will increase to $\langle N_p \rangle\simeq40$. 
\subsection{Effect of Pile Up}
The majority of PU events are low energy QCD processes which will appear as randomly distributed energy in the calorimeter region (about 1 GeV per unit area). This PU can have two detrimental effects for the L1 trigger. Firstly, when the PU is evenly distributing the energy of the real jets is slightly increased thus artificially increasing the total transverse energy. The second problem occurs when the PU is randomly clustered - significantly boosting a single jet or forming fake jets. Due to the non perfect response of the detector and profile of the PU these effects may be dependent on $\eta$. In order to trigger only on hard events the effect of the PU must be mitigated (subtracted). In this report two example methods of pileup subtraction (global $\rho$ and donut) will be discussed.
\subsection{Global $\rho$}
If the detector response is assumed to be approximately $\eta$ independent then one may use an event based PU subtraction. In the case of global $\rho$ first all the jets are reconstructed using the algorithm in \ref{algo}. The jets are then ranked in order of energy density $\rho$ where for jet $i$, $\rho^i$ is defined in equation \ref{rho}.
\begin{equation}
\label{rho}
\rho^i = \frac{p^i_{T}}{A_i} 
\end{equation}
where $A_i$ is the area of the jet. The median $\rho$ ($\rho^m$) of the jets is then taken as the estimator of the PU energy density in the event \cite{jetarea}. All jets are in the event then reduced in energy by $\rho^m \times A^i$ and those with $p_{T}^i < \rho^m \times A_i$ taken to be PU and removed. The correlation of this parameter with the number of interactions is shown in figure **. This method has the advantage of being stable against fluctuations, however, local effects are not considered and there is an inherent under-subtraction bias in the event as the PU jet $p_{T}$ forms a bound on the energy subtracted per jet.
\subsection{Donut Subtraction}

Donut subtraction takes advantage of the area around each jet to make a local subtraction. The energy contribution from the jet is expected to be negligible by the outside ring ($\Delta R^2\simeq 0.2$) \cite{jetmet}. Taking the four strips around the jet as shown in figure \ref{donut} they are ordered in energy and the pile up energy density ($\rho_d$) is taken as the energy of the middle strips divided by their area ($A_d$). This energy density is then subtracted from the jet as for global $\rho$. The dependence of $\rho_d$ on number of vertices is shown in figure **. The highest strip is neglected to remove the contamination of nearby jets and the lowest to account for edge effects. Unlike global, Donut subtraction allows local dependence of pile up to be taken into account, however, it is susceptible to fluctuations as the sampling region is small.

An additional method to remove PU is to require a seed threshold. This takes advantage of the fact that PU jets have flatter topologies as clusters of uniform PU that boosted real jets. A threshold on the seed tower can thus reduce the rate of PU while maintaining efficiency. Donut subtraction may then be performed to correct the remaining jets. A seed threshold is not possible before the global $\rho$ calculation as this relies on lower $p_{T}$ jets for the PU estimation. Optimising the seed (perhaps as a function of $\eta$) is under way, however, a threshold of 2.5GeV (5 level one units) was chosen as a benchmark.
\begin{figure}
\centering
    \includegraphics[width=0.5\textwidth]{Figures/donut}
  \caption{Donut ring for PU estimate. Neglecting the corners the PU energy density is taken as that in the middle two strips.}
  \label{mask}
\end{figure}  
