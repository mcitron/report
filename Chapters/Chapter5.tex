% Chapter Template

\chapter{Phenomenology} % Main chapter title

\label{Chapter5} % Change X to a consecutive number; for referencing this chapter elsewhere, use \ref{ChapterX}

\lhead{Chapter 5. \emph{Phenomenology}} % Change X to a consecutive number; this is for the header on each page - perhaps a shortened title

%----------------------------------------------------------------------------------------
%	SECTION 1
%----------------------------------------------------------------------------------------

\section{Phenomenology}
The experimental data that comes from $\alpha_T$ and other direct and indirect searches may be combined to find the most stringent limits on new physics. This provides a valuable resource for theorists in building models and experimentalists in designing new searches. Mastercode is a leading collaboration involved in making such exclusion planes for constrained GUT models as well as phenomenological models of the MSSM\cite{mcode}. Another approach is to use Natural SUSY (NS) spectra motivated by naturalness and keeping within experimental constraints\cite{joliver}. By combining different searches the limits on these spectra may be shown to be broadly dependent on only the squarks, gluino and LSP. NS spectra thus provide a good measure of general limits on gluinos and squarks. It must be noted, however, that there are special cases such as compressed spectra (where the mass splitting are too small to provide sufficient energies for search sensitivity) to which this will not be applicable.

