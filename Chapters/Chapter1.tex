% Chapter 1

\chapter{Introduction} % 

\label{Chapter1} % For referencing the chapter elsewhere, use \ref{Chapter1} 

\lhead{Chapter 1. \emph{Bounce}} % This is for the header on each page - perhaps a shortened title

%----------------------------------------------------------------------------------------
The Large Hadron Collider (LHC) is a 27 km proton-proton synchrotron. From 2009-2013 it has operated at $\sqrt{s}=7-8TeV$ delivering a maximum instantaneous luminosity, L, of $7 \times 10^{33} cm^{-2}s^{-1}$ with a bunch spacing of $50ns$ \cite{run1}. This has provided around $25fb^{-1}$ of data which has been used to complete the Standard Model (SM) with the discovery of the Higgs boson and placed new models of physics under intense scrutiny \cite{cmshiggs}\cite{atlashiggs}\cite{susyr1}. Supersymmetry is the leading contender to replace the SM. This makes direct searches for SUSY a significant part of the work carried out at the LHC. The search focused on in this report will be $\alpha_T$, an all-hadronic final state search with the Compact Muon Solenoid (CMS) detector which has shown great strength in Run 1 \cite{search}. 

Currently, the LHC and detectors have been shut down for a range of upgrades to allow design conditions to be reached \cite{ls1}. From mid-2015 it is expected that the improvements will allow $\sqrt{s}=13-14TeV$ while increasing the instantaneous luminosity (to a maximum of $1.7\times10^34cm^{-2}s^{-1}$) and halving the bunch spacing \cite{HighLumi}. These upgrades will dramatically improve the reach of searches for high mass particles such as supersymmetric particles (sparticles) as the predicted production probability will be increased \cite{ProjectedCx} . However, the conditions for the CMS detector will be much more challenging.  This will make the operation of the level one (L1) trigger, which must reduce the event rate from $~40Mhz$ to $~100kHz$, far harder. For example, the number of simultaneous interactions (pileup) will increase from around 27 to 40, artificially adding a randomly distributed energy to the event. New hardware and algorithms are required for the L1 trigger to operate effectively in this regime. The work presented in this report concerns jet and pile-up subtraction algorithms for the stage 2 upgrade expected to be implemented in 2016.  










