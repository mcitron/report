% Chapter 1

\chapter{Introduction} % 

\label{Chapter1} % For referencing the chapter elsewhere, use \ref{Chapter1} 

\lhead{Chapter 1. \emph{Bounce}} % This is for the header on each page - perhaps a shortened title

%----------------------------------------------------------------------------------------
The Large Hadron Collider (LHC) is a 27 km proton-proton synchrotron. From 2009-2013 it has operated at $\sqrt{s}=7-8TeV$ delivering a maximum luminosity (L) of $7 \times 10^{33} cm^{-2}s^{-1}$ with a bunch spacing of $50ns$\cite{run1}. This has provided around $25fb^{-1}$ of data which has been used to complete the Standard Model (SM) with the discovery of the Higgs boson and placed new models of physics under intense scrutiny\cite{cmshiggs}\cite{atlashiggs}\cite{susyr1}. Supersymmetry is the leading contender to replace the SM. This makes direct searches for SUSY a significant part of the work carried out at the LHC. The search focused on in this report will be $\alpha_T$, an all-hadronic final state search with the Compact Muon Solenoid (CMS) detector which has shown great strength in Run 1\cite{search}. 

Currently the LHC and detectors have been shut down for a range of upgrades to allow design conditions to be reached\cite{ls1}. From mid-2015 it is expected that the improvements will allow $\sqrt{s}=13-14TeV$ while increasing the luminosity (to a maximum of $1.7\times10^34cm^{-2}s^{-1}$) and halving the bunch spacing\cite{HighLumi}. These upgrades will dramatically improve the reach of searches for high mass particles such as supersymmetric particles (spartcles) as the predicted production probability for supersymmetric particles (sparticles)\cite{ProjectedCx} will be increased. However, the new conditions will make the operation of the L1 trigger, which must reduce the event rate from $~40Mhz$ to $~100kHz$, far harder. For example, the number of simultaneous vertices (pile-up) will increase from around 27 to 40, artificially adding a randomly distributed energy to the event. New hardware and algorithms are required to account for these new conditions. The work presented in this report concerns jet and pile-up subtraction algorithms for the stage 2 upgrade expected to be implemented in 2016.  

%The Level One (L1) Trigger must decide within $\matchcal{O}10ns$ whether to keep or discard each event in order to reduce the data rate from $~40Mhz$ to $~10kHz$. Due to the dominant soft-QCD processes SUSY searches are reliant on the L1 Trigger to maintain efficiencies while reducing background rates. After the upgrade the contribution from soft-QCD will increase   











